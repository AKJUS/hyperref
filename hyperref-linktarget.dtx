% \iffalse meta-comment
%
%% File: hyperref-linktarget.dtx
%
% Copyright (C) 2022-2024 The LaTeX Project
%
% It may be distributed and/or modified under the conditions of the
% LaTeX Project Public License (LPPL), either version 1.3c of this
% license or (at your option) any later version.  The latest version
% of this license is in the file
%
%    http://www.latex-project.org/lppl.txt
%
% This file is part of the "Hyperref bundle" (The Work in LPPL)
% and all files in that bundle must be distributed together.
%
% -----------------------------------------------------------------------
%
% The development version of the bundle can be found at
%
%    https://github.com/latex3/hyperref
%
% for those people who are interested.
%
%<*driver>
\DocumentMetadata{pdfstandard=A-2b}
\documentclass[full]{l3doc}
\usepackage{array,booktabs}
\hypersetup{pdfauthor=The LaTeX Project,
  pdftitle=Make link targets (hyperref bundle)}
\usepackage[skins]{tcolorbox}
\definecolor{myblue}{RGB}{00,33,99}
\newtcolorbox{doccomment}[1][]{enhanced,frame hidden,colback=myblue!10,fontupper=\footnotesize,#1}
%\newtcbox{smalldoccomment}{size=small,on line,enhanced,frame hidden,colback=myblue!10,fontupper=\footnotesize}
\begin{document}
  \DocInput{\jobname.dtx}
\end{document}
%</driver>
% \fi
% \title{^^A
%   Make link targets ^^A
%   \\
%   hyperref bundle
% }
%
% \author{^^A
%  The \LaTeX{} Project\thanks
%    {^^A
%      E-mail:
%        \href{mailto:latex-team@latex-project.org}
%          {latex-team@latex-project.org}^^A
%    }^^A
% }
%
% \date{Version 2024-10-12 v7.01k}
%
% \maketitle
% \begin{documentation}
% \section{Commands to create and adapt targets}
%  This module provides commands to create targets.
%  Their goal is to unify and replace
%  a number of user and internal hyperref commands and to provide a clear
%  interface for package authors and users.
% \subsection{The main command}
% \begin{function}{\MakeLinkTarget}
% \begin{syntax}
% \cs{MakeLinkTarget}\oarg{prefix}\Arg{counter}\\
% \cs{MakeLinkTarget}\oarg{prefix}\{\} \\
% \cs{MakeLinkTarget}*\Arg{manual target}
% \end{syntax}
% \end{function}
%
% \cs{MakeLinkTarget} creates a target for an internal link, (called a destination
% if a PDF is created). In vertical mode the target is created where the command is issued,
% in horizontal mode it is typically (in a PDF) raised by the current \cs{normalbaselineskip}
% (similar to the targets created by \cs{refstepcounter}).
%
% The arguments allow to control the name of the target/anchor/destination. While technically
% any unique name (e.g. some number) would work, names related to the actual counter both
% simplifies the debugging and referencing destinations from external documents. Also \cs{autoref}
% makes use of the name to identify the type of a label: it splits off the part until the first period and
% then looks up for a defined name. So if you create a target name \texttt{ABC.CDE.1.2}, \cs{autoref} tries
% to use \cs{ABCautorefname}.
%
% The name is then stored \emph{globally}\footnote{This means that the deprecated \pkg{hyperref}
% option \texttt{localanchorname} is ignored.} in \cs{@currentHref} and used for example in the
% next \cs{label}.
% A target name must be unique across a document. It is up to the users and package
% authors to ensure this by following the advices given below.
%
% \begin{doccomment}
% A global \cs{@currentHref} means that the target name and
% the label name can get out of sync: if
% for example a numbered equation is used between a section and a \cs{label},
% a reference will show the section number but a link will jump to the equation.
%
% \pkg{hyperref} allows to switch to local assignment of \cs{@currentHref}
% and so to align with the other \cs{label} values with the
% (so-called experimental) option \texttt{localanchorname}. But while on a document
% level this can work, it makes it difficult for package authors
% to add reliable link targets that can be referenced if it is unclear
% if the assignment is local or global.
%
% So the choice was made to set \cs{@currentHref} globally always
% (and to deprecate \texttt{localanchorname}):
% This is the default in \pkg{hyperref} since ever and it didn't lead to
% major problems. It has the advantage to
% allow to put an anchor in a box and move it around. E.g. in the following
% example the two \cs{ref} commands
% reference the section but the second jumps to the top of the rule:
%
% \begin{verbatim}
% \section{An example}\label{sec:title}
% \newpage \raisebox{3cm}[0pt][0pt]{\phantomsection}%
% \rule{1cm}{3cm}\label{sec:page2}
% page two
% \newpage
% \ref{sec:title}, \ref{sec:page2}
% \end{verbatim}
%
% \end{doccomment}
%
%
%
% \subsection*{Target names from counters}
% \begin{syntax}
% \cs{MakeLinkTarget}\oarg{prefix}\Arg{counter}
% \end{syntax}
%
% If the mandatory argument is not empty it is interpreted as a \LaTeX{} counter name.
% The counter is not stepped by the command.
% If the \meta{counter} doesn't exist, a warning is issued and no target is created.
%
% The target name is created as expansion of
% \texttt{\meta{prefix}.\cs{theH\meta{counter}}} (and so both the prefix
% and \cs{theH\meta{counter}} should be expandable).
% The default prefix is the counter name \meta{counter}.
%
% For example:
%
% \medskip
% \noindent
% {\ttfamily
% \begin{tabular}{@{}l@{}ll}
% \cs{MakeLinkTarget}\{section\} &$\Rightarrow$ section.\cs{theHsection} & $\Rightarrow$ section.1.1\\
% \cs{MakeLinkTarget}[sec]\{section\} &$\Rightarrow$ sec.\cs{theHsection}
% \end{tabular}
% }
% \medskip
%
% \cs{theH\meta{counter}} must be defined:
% \cs{MakeLinkTarget} does not try like other \pkg{hyperref} commands to
% construct the command on-the-fly but warns and creates no
% target if it doesn't exist.
% Be aware that if \pkg{hyperref} is used with the option \texttt{implicit=false} it
% does not predefine and create \cs{theH\meta{counter}} representations
% so targets could be missing!\footnote{It is planed that \LaTeX{} directly defines
% the \cs{theH\meta{counter}} for all counters created with \cs{newcounter}.}
%
%
% Typically \cs{theH\meta{counter}} should expand to numbers
% and periods and make use of parent counters to give a unique representation.
% The prefix can be a more or less arbitrary string. Spaces are allowed (but not
% really recommended).
% A star at the end should be avoided to prevent clashes with
% the names created when the internal counter is used.
% The use of non-ASCII chars with pdflatex depends on the \LaTeX{} version: With newer
% version many of them are safe and with a format 2022-06-01 or newer
% it should be even possible to use chars in a target name which
% are undeclared and can't be typeset in the document.
% But despite the fact that it works, it is recommended to stick to ASCII
% and to avoid spaces: It is unclear if all PDF viewers and editors can handle
% them, also they give rather unreadable names in the PDF like |(\360\237\246\206)|
% and so make debugging harder.
%
%
% Using a special prefix can be useful if the actual counter
% has an internal name like e.g. \texttt{tcb@cnt@example}.
% \texttt{\cs{MakeLinkTarget}[tcbexample]\{tcb@cnt@example\}} then gives
% a nicer looking target name. Care should be taken not to clash
% with existing counter names and names for \cs{autoref}
% should be adjusted if needed.
%
%
% \subsection*{Targets using the internal counter}
% \begin{syntax}
% \cs{MakeLinkTarget}\oarg{prefix}\{\}
% \end{syntax}
%
% If the mandatory argument is empty a target name based on
% an internal absolute counter is created.
% The counter is stepped at every call (and also by other \pkg{hyperref} commands).
% The prefix is added with a star and a period (see above for
% the allowed chars). The default prefix is \texttt{page}.
%
%
% \begin{syntax}
% \begin{tabular}{@{}ll}
% \cs{MakeLinkTarget}\texttt{\{\}} &$\Rightarrow$ \texttt{page*.1}\\
% \cs{MakeLinkTarget}\texttt{\{\}} &$\Rightarrow$ \texttt{page*.2}\\
% \cs{MakeLinkTarget}[section]\texttt{\{\}} &$\Rightarrow$ \texttt{section*.3}\\
% \end{tabular}
% \end{syntax}
%
% As an example the
% \pkg{hyperref} command \cs{phantomsection} is equivalent to
% \texttt{\cs{MakeLinkTarget}[section]\{\}}.
%
% \subsection*{Manual target names}
% \begin{syntax}
% \cs{MakeLinkTarget}\texttt{*}\Arg{manual target name}
% \end{syntax}
%
% If the starred variant is used a manual target name is created.
%
% \texttt{\cs{MakeLinkTarget}*\{destname\}} is roughly equivalent to
% \texttt{\cs{hypertarget}\{destname\}\{\}} but unlike the
% latter it also raises the destination in a PDF, there is no text argument,
% and---most importantly---it updates \cs{@currentHref}, that means the target
% name is stored by the next \cs{label}, something that \cs{hypertarget} doesn't do).
% The target name is expanded, see the remarks about prefixes above regarding
% the allowed chars.
%
% When creating manually targets care should be taken to avoid
% clashes with automatic target names.
% As all automatic targets contain at least one period,
% names without a period are recommended.
%
% \begin{doccomment}
% The second (text) argument of \cs{hypertarget} is only used
% if the option \texttt{nesting} is set to true---but this option isn't used
% anywhere (in PDF the idea of a text in the anchor or nesting of anchors
% makes no sense anyway but also when html is produced e.g. with \texttt{make4ht}
% it is not used). So probably the option will be deprecated and removed.
% \end{doccomment}
%
% \subsection{Manipulate the next target name}
%
% Targets are sometimes created in places where it is difficult to inject
% a label to retrieve the target name for use e.g. in a bookmark.
% The next command allows to change the next target name:
%
% \begin{function}{\NextLinkTarget}
% \begin{syntax}
% \cs{NextLinkTarget}\Arg{manual target}
% \end{syntax}
%
% The command changes the next target name to \meta{manual target}.
% It does the same as the \cs{hypersetup} key
% \texttt{next-anchor} and also affects targets created by \cs{refstepcounter}.
%\end{function}
%
% A use case are bookmarks for the table of contents
% where you know that the heading will create
% a target:
%
% \begin{verbatim}
% \documentclass{book}
% \usepackage{bookmark}
%
% \begin{document}
% \bookmark[dest=toc]{Table of Contents}
% \NextLinkTarget{toc}
% \tableofcontents
% \chapter{A}
% \end{document}
% \end{verbatim}
%
%
% \subsection*{Hooks}
%
% \begin{function}{makelinktarget}
% The hook \texttt{makelinktarget}\footnote{The hook uses a plain name
% without reference to the \pkg{hyperref} package in anticipation of the move
% of this code into the \LaTeX{} kernel.}
% is executed at the begin of the commands. It is inside a group and
% so can be used to locally change settings. See below for an example.
% \end{function}
%
%
% \subsection*{Suppressing the target}
%
% \begin{function}{\LinkTargetOn,\LinkTargetOff}
% \begin{syntax}
% \cs{LinkTargetOn}\\
% \cs{LinkTargetOff}
% \end{syntax}
% \end{function}
%
%
% This commands allows to switch on and off locally the creation of a target with
% \cs{MakeLinkTarget}.
% The switches are also honored by \cs{refstepcounter}\footnote{currently
% \cs{refstepcounter} doesn't use \cs{MakeLinkTarget} itself but this
% will probably change.}.
% This allows to suppress the target
% from an internal \cs{refstepcounter} and replace it by some manual version
% by using grouping:
%
% \begin{verbatim}
% \LinkTargetOff %suppress anchor in internal refstepcounter
% ...
% \refstepcounter{...}
% ...
% {\LinkTargetOn\MakeLinkTarget*{mytarget}} %create manual anchor
% ...
% \LinkTargetOn
% \end{verbatim}
%
%
% \subsection*{Raising the target}
%
% In horizontal mode the target is raised by the current value of \cs{normalbaselineskip}.
%
% To change this the hook can be used e.g. to double the value everywhere:
%
% \begin{verbatim}
% \AddToHook{cmd/MakeLinkTarget/before}
%   {\setlength\normalbaselineskip{2\normalbaselineskip}}
% \leavevmode\MakeLinkTarget{section}
% \end{verbatim}
%
%
% \subsection{Changing all target names}
%
% \begin{function}{\SetLinkTargetFilter}
% \begin{syntax}
% \cs{SetLinkTargetFilter}\Arg{filter code using \#1}
% \end{syntax}
% \end{function}
%
% \pkg{hyperref} provides the command \cs{HyperDestNameFilter} to change all target names.
% It is applied\footnote{In the backend code, so it depends actually on
% the driver if it is honored or not}
% to every target name and is also used in references,
% but it doesn't change \cs{@currentHref} itself. So after
%
% \begin{verbatim}
% \renewcommand*{\HyperDestNameFilter}[1]{docA-#1}
% \end{verbatim}
%
% you would get in the PDF everywhere the prefix \texttt{docA}
%
% \begin{verbatim}
% %destination names:
% <<
% /Names [(docA-Doc-Start) 7 0 R (docA-chapter.1) 8 0 R (docA-page.1) 6 0 R]
% /Limits [(docA-Doc-Start) (docA-page.1)]
% >>
% %link to a chapter
% /A << /S /GoTo /D (docA-chapter.1) >>
% %link from the bookmark
% << /S /GoTo /D (docA-chapter.1) >>
% \end{verbatim}
%
% but the label info in the \texttt{.aux} would show only \texttt{chapter.1}:
%
% \begin{verbatim}
% \newlabel{chap}{{1}{1}{Title}{chapter.1}{}}
% \end{verbatim}
%
% and so \cs{autoref} is still able to extract the counter name.
%
% \cs{MakeLinkTarget} uses this filter too: it would break internal link
% commands if it would ignore it.
% To stay compatible with future development the filter should not be redefined
% directly but be set with \cs{SetLinkTargetFilter}.
% The command can only be used in the preamble.
%
% \begin{verbatim}
% \SetLinkTargetFilter{docA-#1}
% \end{verbatim}
% \end{documentation}
%
% \begin{implementation}
% \section{Implementation}
%    \begin{macrocode}
%<@@=hyp>
%<*header>
\ProvidesExplPackage{hyperref-linktarget}{2022-06-20}{v7.00s}
  {Making targets, destinations and anchors -- module of hyperref}
%</header>
%<package-include>\ExplSyntaxOn
%    \end{macrocode}
% \subsection{Variables}
%    \begin{macrocode}
%<*package>
%    \end{macrocode}
% \begin{macro}{\l_@@_target_create_bool}
%  This boolean decides if a target is created at all.
%  (it will replace \cs{@skiphyperref} long term)
%    \begin{macrocode}
\bool_new:N \l_@@_target_create_bool
\bool_set_true:N \l_@@_target_create_bool
%    \end{macrocode}
% \end{macro}
% \begin{macro}{makelinktarget}
% This hook is used to adapt for example the raising
%    \begin{macrocode}
\hook_new:n {makelinktarget}
%    \end{macrocode}
% \end{macro}
% \subsection{Helper commands}
% \begin{macro}{\_@@_target_raise:n}
% We need a command to raise the targets.
% It is mostly a copy from the hyperref command
% but we removed the hooks and use \cs{normalbaselineskip}.
% TODO: The code to save/restore the space factor should
% be replaced by kernel methods.
%    \begin{macrocode}
\cs_new_protected:Npn \@@_target_raise:n #1
 {
   \mode_if_vertical:TF
    { #1 }
    {
     \Hy@SaveSpaceFactor
     \penalty\@M
     \smash
      {
       \box_move_up:nn
        { \normalbaselineskip }
        {
         \hbox:n
          {
           \Hy@RestoreSpaceFactor
           #1
           \Hy@SaveSpaceFactor
          }
        }
      }
     \Hy@RestoreSpaceFactor
    }
 }
%    \end{macrocode}
% \end{macro}
% \subsection{Providing the commands}
% In anticipation of the addition of the main commands to the
% kernel as no-ops we provide them:
%    \begin{macrocode}
\ProvideDocumentCommand\LinkTargetOn{}{}
\ProvideDocumentCommand\LinkTargetOff{}{}
\ProvideDocumentCommand\MakeLinkTarget{sO{}m}{}
\ProvideDocumentCommand\NextLinkTarget{m}{}
%    \end{macrocode}
%
% \subsection{Target on and off switch}
% \begin{macro}{\LinkTargetOn,\LinkTargetOff}
%    \begin{macrocode}
\RenewDocumentCommand\LinkTargetOn {}
 {
   \bool_set_true:N \l_@@_target_create_bool
 }


\RenewDocumentCommand\LinkTargetOff {}
 {
   \bool_set_false:N \l_@@_target_create_bool
 }
%    \end{macrocode}
% \end{macro}
%
% \begin{macro}{\MakeLinkTarget}
% This is the main command. To keep it simple
% we allow an optional argument also for the manual
% command but ignore it for now.
%    \begin{macrocode}
\RenewDocumentCommand\MakeLinkTarget {s O{} m}
  {
    \bool_if:NT \l_@@_target_create_bool
      {
        \group_begin:
        \hook_use:n { makelinktarget }
        \IfBooleanTF  {#1}
          {
            \@@_target_manual:nn {#2}{#3}
          }
          {
            \@@_target_counter:nn {#2}{#3}
          }
        \group_end:
      }
  }
%    \end{macrocode}
% \end{macro}
% \begin{macro}{\@@_target_manual:nn}
% This is the code for the manual target name. The prefix is simply ignored.
%    \begin{macrocode}
\cs_new_protected:Npn  \@@_target_manual:nn #1 #2 %#1 prefix, #2 name
  {
    \tl_gset:Ne \@currentHref {#2}
    \hook_use:n {__hyp/target/setname}
    \@onelevel@sanitize\@currentHref
    \@@_target_raise:n {\hyper@anchorstart{\@currentHref}\hyper@anchorend}
  }
%    \end{macrocode}
% \end{macro}
% \begin{macro}{\@@_target_counter:nn}
% The code for counter related targets must be split into the case
% where the internal counter is used, and where a user counter is used
%    \begin{macrocode}
\cs_new_protected:Npn \@@_target_counter:nn #1 #2  %#1 prefix, #2 counter or empty
  {
    \tl_if_blank:nTF {#2}
      {
        \@@_target_counter_anon:n {#1}
      }
      {
        \@@_target_counter_doc:nn {#1}{#2}
      }
  }

%    \end{macrocode}
%\end{macro}
%\begin{macro}{\@@_target_counter_anon:n}
% This creates the target with the internal count.
% We use the same (tex) count \cs{Hy@linkcounter} than the other
% hyperref commands.
%    \begin{macrocode}
\cs_new_protected:Npn \@@_target_counter_anon:n #1
  {
    \int_gincr:N\Hy@linkcounter
    \tl_gset:Ne \@currentHref
      {\tl_if_blank:nTF{#1}{page}{#1}*.\int_use:N\Hy@linkcounter}
    \hook_use:n {__hyp/target/setname}
    \@onelevel@sanitize\@currentHref
    \@@_target_raise:n {\hyper@anchorstart{\@currentHref}\hyper@anchorend}
  }
%    \end{macrocode}
% \end{macro}

% \begin{macro}{\@@_target_counter_doc:nn}
% And now the target with the user counter. We warn if
% the counter or the representation doesn't exist
%    \begin{macrocode}
\cs_new_protected:Npn \@@_target_counter_doc:nn #1 #2
  {
    \bool_lazy_or:nnTF { \cs_if_free_p:c {c@#2} } { \cs_if_free_p:c {theH#2} }
      {
        \PackageWarning {hyperref}{Counter~'#2'~or~the~representation~'\string\theH#2`\MessageBreak
        don't~exist.~No~target~created.}{}
      }
      {
        \tl_gset:Ne \@currentHref {\tl_if_blank:nTF{#1}{#2}{#1}.\use:c{theH#2}}
        \hook_use:n {@@/target/setname}
        \@onelevel@sanitize\@currentHref
        \@@_target_raise:n {\hyper@anchorstart{\@currentHref}\hyper@anchorend}
      }
  }
%    \end{macrocode}
% \end{macro}
%
% \begin{macro}{\NextLinkTarget}
%  we rely on the internal hook to set the next target name:
%    \begin{macrocode}
\RenewDocumentCommand\NextLinkTarget {m}
  {
    \hook_gput_next_code:nn {@@/target/setname}
      {
        \tl_gset:Ne \@currentHref {#1}
      }
  }

%    \end{macrocode}
% \end{macro}
% \begin{macro}{\SetLinkTargetFilter}
% This is an interface to \cs{HyperDestNameFilter}
%    \begin{macrocode}
\NewDocumentCommand\SetLinkTargetFilter {m}
  {
    \cs_set:Npn \HyperDestNameFilter ##1 {#1}
  }
\@onlypreamble \SetLinkTargetFilter
%    \end{macrocode}
% \end{macro}
%
%    \begin{macrocode}
%</package>
%<@@=>.
%<package-include>\ExplSyntaxOff
%    \end{macrocode}
%\end{implementation}
% \Finale
\endinput
