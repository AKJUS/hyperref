% \iffalse
%% File: nameref.dtx
%% Copyright (C) 1995-1999 Sebastian Rahtz
%%               2000 Sebastian Rahtz, Heiko Oberdiek
%%               2001-2012 Heiko Oberdiek
%%               2016-2019 Oberdiek Package Suport Group
%%               2019-2023 The LaTeX Project
%%      https://github.com/latex3/hyperref/issues
%%
%% This file is part of the `Hyperref Bundle'.
%% -------------------------------------------
%%
%% This work may be distributed and/or modified under the
%% conditions of the LaTeX Project Public License, either version 1.3
%% of this license or (at your option) any later version.
%% The latest version of this license is in
%%   http://www.latex-project.org/lppl.txt
%% and version 1.3 or later is part of all distributions of LaTeX
%% version 2005/12/01 or later.
%%
%% This work has the LPPL maintenance status `maintained'.
%%
%% The Current Maintainer of this work is The LaTeX Project.
%%
%% The list of all files belonging to the `Hyperref Bundle' is
%% given in the file `manifest.txt'.
%
%<*dtx>
\ProvidesFile{nameref.dtx}
%</dtx>
%<package>\NeedsTeXFormat{LaTeX2e}
%<package>\ProvidesPackage{nameref}
%<driver>\ProvidesFile{nameref.drv}
% \fi
% \ProvidesFile{nameref.dtx}
  [2022-05-17 v2.50 Cross-referencing by name of section]%
%
%
% \iffalse
%<*driver>
\PassOptionsToPackage{numbered}{hypdoc}
\documentclass{ltxdoc}
\usepackage[numbered]{hypdoc}
\usepackage{bmhydoc}
\usepackage{hologo}
\EnableCrossrefs
\CodelineIndex
\newcommand*{\xpackage}[1]{\textsf{#1}}
\begin{document}
  \GetFileInfo{nameref.sty}%
  \title{%
    Section name references in \LaTeX
    \thanks{%
      This file has version number \fileversion, %
      last revised \filedate.%
    }%
  }%
  \author{Sebastian Rahtz (deceased)\\%
    Heiko Oberdiek (inactive)\\%
    The \LaTeX\ Project\\%
    \texttt{https://github.com/latex3/hyperref/issues}}%
  \date{\filedate}%
  \maketitle
  \makeatletter
  \@ifundefined{HyperrefOverview}{}{%
    \HyperrefOverview*%
  }%
  \tableofcontents
  \DocInput{nameref.dtx}%
  \PrintIndex
\end{document}
%</driver>
% \fi
%
% \StopEventually{}
%
% \section{Introduction}
% Cross-referencing to include the \emph{name} of the section, rather than
% just the number or page. This works by redefining some of the inside
% \LaTeX\ macros, so it is vulnerable to future changes.
%
% \section{Background and advice for package and class authors}
% The package redefines the \cs{label} command to write additionally
% \cs{@currentlabelname} to the \texttt{.aux} file. This can then be referenced with
% \cs{nameref}. Package authors who want to support \textsf{nameref} should
% define this command (locally) in a suitable way.
% The content should be so that it can be safely written
% to the \texttt{.aux} file. \textsf{nameref} itself uses the package
% \textsf{gettitlestring} but this is not required.
%
% \textsf{nameref} patches a number of commands (mostly sectioning
% commands) and environments so that they
% store \cs{@currentlabelname}. A class or a package that want to provide native
% support instead can suppress this patches by defining commands described below.
% The commands must be defined before the patch is made, in most cases before
% \textsf{nameref} is loaded, sometimes only before begin document.
%
% \begin{description}
% \item[sectioning commands] Redefined are
%  \begin{itemize}
%  \item
%  \cs{@sect}: beside adding a command that stores the
%  title of a sectioning command (if present the optional argument is used)
%  \textsf{nameref} also stores the level number in a counter \texttt{section@level}
%  (reason is unknown, only two packages, \textsf{e-french} and \textsf{pdfscreen}
%  seem to use it), and  surrounds the text with \cs{Sectionformat} (only used by
%  \textsf{acmart}, \textsf{memoir}, \textsf{pdfscreen}, \textsf{pdfslide}), a
%  command which takes two argument: the first the text, the second the level.
%
%  \item \cs{@ssect}: command to store the title added,
%  \cs{Sectionformat} is added.
%
%  \item \cs{M@sect} (memoir class): command to store the title added
%
%  \item \cs{@part}, \cs{@spart}, \cs{@chapter}, \cs{@schapter}:
%  command to store the title added
%  \item \cs{ttl@sect@i} (from the \textsf{titlesec} package):
%   command to store the title added
%  \end{itemize}
%
% All the patches excepted the last are disabled if the command
% \cs{NR@nopatch@sectioning} is defined (the actual definition is irrelevant, only
% existence is tested. The titlesec patch is disabled if
% \cs{NR@nopatch@titlesec} is defined.
% This currently only suppress the actual patches, the counter,
% \cs{Sectionformat} and also the internal copies \cs{NR@sect} etc are still
% defined.
%
% \item[captions] The following commands are patched
% \begin{itemize}
% \item \cs{@caption}: command to store the title added. This is done at begin
% document, but only if \textsf{hyperref} is not detected. With \textsf{hyperref}
% the patch is done there (and must be disabled in \textsf{hyperref}).
%
% This patch can be disabled by defining \cs{NR@nopatch@caption}
%
% \item \cs{LT@c@ption} (from the \textsf{longtable} package):
% command to store the title added.
%
% This patch can be disabled by defining \cs{NR@nopatch@longtable}
%
% \item \cs{lst@MakeCaption} (from the \textsf{listings} package)
%
% This patch can be disabled by defining \cs{NR@nopatch@listings}
% \end{itemize}
%
% \item[lists] The following command is patched
% \begin{itemize}
% \item \cs{descriptionlabel}: command to store the label text added. To
% reference this text the \cs{label} must be put \emph{inside} the label.
% This is not compatible with the \textsf{enumitem} package as internal
% formatting commands are written to the \texttt{.aux} file too.
%
% This patch can be disabled by defining \cs{NR@nopatch@lists}
% \end{itemize}
%
% \item[theorems]
%
% \begin{itemize}
% \item \cs{@opargbegintheorem}: command to store the title added.
%
% This patch can be disabled by defining \cs{NR@nopatch@theorem}
%
% \item \cs{@begintheorem} (if amsthm is loaded and if
% the command has a tested meaning): command to store the title added.
%
% This patch can be disabled by defining \cs{NR@nopatch@amsthm}
%
% \end{itemize}
% \item[beamer] In the \textsf{beamer} class the following commands are
% patched: \cs{beamer@section}, \cs{beamer@subsection}, \cs{beamer@subsubsection},
% \cs{beamer@descriptionitem} and the template \texttt{theorem begin}. If
% \cs{NR@nopatch@beamer} is defined nothing is done.
% \end{description}
%
% \section{History}
% Sebastian Rahtz, for Lou Burnard, March 15th 1994;\\
% SPQR CERN July 1994;\\
% Last mod. Sept. 3th MG;\\
% Sept. 19th SPQR;\\
% April 11th 1995 SPQR (added section formatting hook);\\
% April 14th 1995 SPQR (compatibility with hyperref);\\
% June 22 1995 SPQR (removed typeout from Sectionformat);\\
% September 22 1997 added varioref hacks by Corey Minyard;\\
% September 28th changes by David Carlisle.\\
% More recent changes are recorded in ChangeLog.
%
% The versions before 2009/11/27 v2.32 did not expand the
% title strings (of \cs{section}, \cs{caption}, \dots).
% Then the behaviour switched to expanding, because this
% allows to remove unwanted macros (\cs{label}, \dots).
% Despite this method is used in package \textsf{titleref}
% and class \textsf{memoir}, the first feedbacks let me conclude
% that this is not robust enough. Therefore I wrote package
% \textsf{gettitlestring} that uses the non-expanding
% method as default. It also allows me to use this package
% in package \textsf{zref-titleref}.
% See the package documentation of package \textsf{gettitlestring}
% how to change the method. It also provides a hook to remove
% additional unwanted macros, not supported by the package.
% Example:
%\begin{quote}
%\begin{verbatim}
%\usepackage{gettitlestring}
%\GetTitleStringDisableCommands{%
%  \let\mylabel\@gobble
%  \renewcommand\myindex[2]{}%
%}
%\end{verbatim}
%\end{quote}
%
% \section{Implementation}
%    \begin{macrocode}
%<*package>
%    \end{macrocode}
%
%    \begin{macrocode}
\RequirePackage{refcount}[2006/02/12]
\RequirePackage{gettitlestring}[2009/12/18]
\RequirePackage{ltxcmds}[2009/12/12]
%    \end{macrocode}
%
% In formats older than 2023-06-01 
% we redefine |\label| so that it also writes the name of the
% current section to the .aux file; if the name ends in a dot,
% we zap it. To allow for the \xpackage{hyperref} package, also write
% fourth and fifth fields (empty by default;
% used for cross-ref name, and file).
%    \begin{macro}{\NR@strip@period}
%    \begin{macrocode}
\def\NR@strip@period#1.\ltx@empty#2\@nil{#1}
%    \end{macrocode}
%    \end{macro}
%
%    \begin{macro}{\strip@period}
% Macro \cs{strip@period} was removed in \xpackage{nameref}
% 2012/07/28 v2.41 and reinserted in 2012/07/31 v2.42 for compatibilty
% reasons with package \xpackage{subfig}.
%    \begin{macrocode}
\def\strip@period#1.\relax#2\@@@{#1}
%    \end{macrocode}
%    \end{macro}
%
%    \begin{macrocode}
\def\NR@sanitize@labelname{%
  \@onelevel@sanitize\@currentlabelname
  \edef\@currentlabelname{%
    \expandafter\NR@strip@period\@currentlabelname
    \ltx@empty.\ltx@empty\@nil
  }%
}
%    \end{macrocode}
%
% \subsection{\cs{label} stuff}
%
%    Here is tested, whether package \xpackage{showkeys} is loaded.
%    But if option final is used, the macros of \xpackage{showkeys}
%    are not defined, so we have to check an additional
%    test.
%    \begin{macro}{\ifNR@showkeys}
%    \begin{macrocode}
\newif\ifNR@showkeys
\NR@showkeysfalse
\@ifpackageloaded{showkeys}{%
  \begingroup
    \@ifundefined{SK@@label}{%
    }{%
      \global\NR@showkeystrue
    }%
  \endgroup
}{}
%    \end{macrocode}
%    \end{macro}   
%    \begin{macro}{\label@hook}
% Starting with LaTeX 2023-06-01 it is not longer needed to 
% redefine \cs{label} as the kernel provides a hook and  
% stores \cs{@currentlabelname}. We still support
% \cs{label@hook} for now as it is used below in \cs{slabel}
% and also in hyperref.    
%    \begin{macrocode}
\providecommand*{\label@hook}{}
\ltx@GlobalAppendToMacro{\label@hook}{%
    \NR@sanitize@labelname
    }
\@ifl@t@r\fmtversion{2023-06-01}
  {
    \AddToHookWithArguments{label}{\def\label@name{#1}\label@hook}
  }
  {
    
%    \end{macrocode}
%    \end{macro}
%    \begin{macro}{\label}
%    \begin{macrocode}
    \def\label#1{%
      \@bsphack
      \begingroup
        \def\label@name{#1}%
        \label@hook
        \protected@write\@auxout{}{%
          \string\newlabel{#1}{%
            {\@currentlabel}%
            {\thepage}%
            {\@currentlabelname}%
            {\@currentHref}{}%
          }%
        }%
      \endgroup
      \@esphack
    }%
    \ifNR@showkeys
      \def\label#1{%
        \@bsphack
        \SK@\SK@@label{#1}%
        \begingroup
          \def\label@name{#1}%
          \label@hook
          \protected@write\@auxout{}{%
            \string\newlabel{#1}{%
              {\@currentlabel}%
              {\thepage}%
              {\@currentlabelname}%
              {\@currentHref}{}%
            }%
          }%
        \endgroup
        \@esphack
      }%
    \fi
  }
%    \end{macrocode}
%    \end{macro}
%    \begin{macro}{\ltx@label}
%  Starting with LaTeX 2023-06-01 it is not longer needed to update the
%  amsmath command \cs{ltx@label} as nameref no longer changes \cs{label} 
%    \begin{macrocode}
\@ifl@t@r\fmtversion{2023-06-01}
  {}{\let\ltx@label\label}
%    \end{macrocode}
%    \end{macro}
% Needed for the \emph{subeqnarray} package.
%    \begin{macro}{\slabel}
%    \begin{macrocode}
\@ifundefined{slabel}{}{%
  \def\slabel#1{%
    \@bsphack
    \if@filesw
      \begingroup
        \def\label@name{#1}%
        \label@hook
        \let\thepage\relax
        \def\protect{\noexpand\noexpand\noexpand}%
        \edef\@tempa{%
          \write\@auxout{%
            \string\newlabel{#1}{%
              {\thesubequation}%
              {\thepage}%
              {\@currentlabelname}%
              {\@currentHref}{}%
            }%
          }%
        }%
      \expandafter\endgroup
      \@tempa
      \if@nobreak\ifvmode\nobreak\fi\fi
    \fi
    \@esphack
  }%
}
%    \end{macrocode}
%    \end{macro}
% Overload an AMS \hologo{LaTeX} command, which uses |\newlabel|. Sigh!
%    \begin{macro}{\@writetocindents@}
%    \begin{macrocode}
\def\@writetocindents@{%
  \begingroup
  \@for\@tempa:=-1,0,1,2,3\do{%
    \immediate\write\@auxout{%
      \string\newlabel{tocindent\@tempa}{%
        \csname r@tocindent\@tempa\endcsname{}{}{}{}%
      }%
    }%
  }%
  \endgroup
}
%    \end{macrocode}
%    \end{macro}
%
% \subsection{Changes of section macros}
%
% Add to the underlying section heading macros so that they
% note the section name for use by label.
%
% If a section heading or the like has a |\label| in it,
% we need to extract it, or subsequent processing breaks.
%
%    \begin{macro}{\NR@gettitle}
%    \begin{macrocode}
\def\NR@gettitle#1{%
  \GetTitleString{#1}%
  \let\@currentlabelname\GetTitleStringResult
}
%    \end{macrocode}
%    \end{macro}
%
% First the numbered sections. While we are about it, put in
% a useful section formatting macro.
%    \begin{macro}{\NR@sect}
%    \begin{macrocode}
\let\NR@sect\@sect
%    \end{macrocode}
%    \end{macro}
%    \begin{macrocode}
\newcounter{section@level}
%    \end{macrocode}
%    \begin{macro}{\@sect}
%    \begin{macrocode}
\@ifundefined{NR@nopatch@sectioning}
 {\def\@sect#1#2#3#4#5#6[#7]#8{%
   \setcounter{section@level}{#2}%
   \NR@gettitle{#7}%
   \NR@sect{#1}{#2}{#3}{#4}{#5}{#6}[{#7}]{\Sectionformat{#8}{#2}}%
 }}{}
%    \end{macrocode}
%    \end{macro}
% and now the unnumbered ones
%    \begin{macro}{\NR@ssect}
%    \begin{macrocode}
\let\NR@ssect\@ssect
%    \end{macrocode}
%    \end{macro}
%    \begin{macro}{\@ssect}
%    \begin{macrocode}
\@ifundefined{NR@nopatch@sectioning}
 {\def\@ssect#1#2#3#4#5{%
   \NR@gettitle{#5}%
   \NR@ssect{#1}{#2}{#3}{#4}{\Sectionformat{#5}{#1}}%
 }}{}
%    \end{macrocode}
%    \end{macro}
%
% Support for class memoir.
%
%    \begin{macrocode}
\@ifundefined{NR@nopatch@sectioning}
 {\ltx@IfUndefined{ifheadnameref}{%
  }{%
    \ltx@IfUndefined{M@sect}{%
    }{%
      \let\NRorg@M@sect\M@sect
      \def\M@sect#1#2#3#4#5#6[#7][#8]{%
        \ifheadnameref
          \NR@gettitle{#8}%
        \else
          \NR@gettitle{#7}%
        \fi
        \NRorg@M@sect{#1}{#2}{#3}{#4}{#5}{#6}[{#7}][{#8}]%
      }%
    }%
  }}{}
%    \end{macrocode}
%
% Parts and chapters are treated differently. sigh.
% \cs{@part} and \cs{@spart} are \cs{long} because of class memoir.
%
%    \begin{macrocode}
\let\NR@part\@part
\let\NR@spart\@spart
\let\NR@chapter\@chapter
\let\NR@schapter\@schapter
\@ifundefined{NR@nopatch@sectioning}
 {%
  \long\def\@part[#1]#2{%
    \NR@gettitle{#1}%
    \NR@part[{#1}]{#2}%
  }
  \long\def\@spart#1{%
    \NR@gettitle{#1}%
    \NR@spart{#1}%
  }
  \def\@chapter[#1]#2{%
    \NR@gettitle{#1}%
    \NR@chapter[{#1}]{#2}%
  }
  \@ifclassloaded{memoir}{%
    \ltx@IfUndefined{ifheadnameref}{%
    }{%
      \def\@chapter[#1]#2{%
        \ltx@IfUndefined{ch@pt@c}{%
          \NR@gettitle{#1}%
        }{%
          \ifx\ch@pt@c\ltx@empty
            \NR@gettitle{#2}%
          \else
            \def\NR@temp{#1}%
            \ifx\NR@temp\ltx@empty
              \expandafter\NR@gettitle\expandafter{\ch@pt@c}%
            \else
              \ifheadnameref
                \NR@gettitle{#1}%
              \else
                \expandafter\NR@gettitle\expandafter{\ch@pt@c}%
              \fi
            \fi
          \fi
        }%
        \NR@chapter[{#1}]{#2}%
      }%
    }%
  }{}
  \def\@schapter#1{%
    \NR@gettitle{#1}%
    \NR@schapter{#1}%
  }}{}
%    \end{macrocode}
%
% Captions
%
%    \begin{macrocode}
\AtBeginDocument{%
  \@ifpackageloaded{hyperref}{}
  {%
    \@ifundefined{NR@nopatch@caption}
     {\let\NR@@caption\@caption
      \long\def\@caption#1[#2]{%
      \NR@gettitle{#2}%
      \NR@@caption{#1}[{#2}]%
       }}%
     {}%
  }%
%    \end{macrocode}
%
%    \begin{macrocode}
  \@ifpackageloaded{longtable}{%
   \@ifundefined{NR@nopatch@longtable}
    {%
      \let\NRorg@LT@c@ption\LT@c@ption
      \def\LT@c@ption#1[#2]#3{%
       \NRorg@LT@c@ption{#1}[{#2}]{#3}%
       \def\@tempa{#2}%
       \ifx\@tempa\@empty
       \else
        \NR@gettitle{#2}%
       \fi
      }%
    }{}%
  }{}%
%    \end{macrocode}
%
% Environment `description'.
%    \begin{macrocode}
  \ltx@IfUndefined{descriptionlabel}{%
  }{%
    \@ifundefined{NR@nopatch@lists}
     {\let\NRorg@descriptionlabel\descriptionlabel
      \def\descriptionlabel#1{%
       \NR@gettitle{#1}%
       \NRorg@descriptionlabel{#1}%
       }%
     }{}%
  }%
}
%    \end{macrocode}
%
% Compatibility for package titlesec.
%    \begin{macrocode}
\AtBeginDocument{%
  \@ifpackageloaded{titlesec}{%
    \let\NRorg@ttl@sect@i\ttl@sect@i
    \@ifundefined{NR@nopatch@titlesec}
     {%
       \def\ttl@sect@i#1#2[#3]#4{%
         \NR@gettitle{#4}%
         \NRorg@ttl@sect@i{#1}{#2}[{#3}]{#4}%
         }%
      }{}%
  }{}%
}
%    \end{macrocode}
%
% \subsubsection{Class beamer}
%
%    \begin{macrocode}
\@ifclassloaded{beamer}{%
 \@ifclasslater{beamer}{2022-05-17}
  {}%do nothing
  {%
    \@ifundefined{NR@nopatch@beamer}
    {%
      \def\NR@PatchBeamer#1{%
        \expandafter\let\csname NRorg@beamer@#1\expandafter\endcsname
                        \csname beamer@#1\endcsname
        \begingroup
          \edef\x{\endgroup
            \long\def\expandafter\noexpand\csname beamer@#1\endcsname
                [####1]{% hash-ok
              \noexpand\NR@gettitle{####1}%
              \expandafter\noexpand\csname NRorg@beamer@#1\endcsname
              [{####1}]%
            }%
          }%
        \x
      }%
      \AddToHook{class/beamer/after}{%
      \NR@PatchBeamer{section}%
      \NR@PatchBeamer{subsection}%
      \NR@PatchBeamer{subsubsection}%
%    \end{macrocode}
%
%    Theorem names.
%    \begin{macrocode}
      \addtobeamertemplate{theorem begin}{%
       \expandafter\NR@gettitle\expandafter{\inserttheoremaddition}%
      }{}%
%    \end{macrocode}
%    \begin{macrocode}
      \@ifdefinable{\NRorg@beamer@descriptionitem}{%
        \let\NRorg@beamer@descriptionitem\beamer@descriptionitem
        \renewcommand{\beamer@descriptionitem}[1]{%
          \NR@gettitle{#1}%
          \NRorg@beamer@descriptionitem{#1}%
        }%
      }%
     }%Hook
   }{}%@ifundefined
  }%class later
 }{}
%    \end{macrocode}
%
% \subsubsection{Theorems}
%
%    \begin{macrocode}
\ltx@IfUndefined{@opargbegintheorem}{}{%
  \let\NRorg@opargbegintheorem\@opargbegintheorem
  \def\@opargbegintheorem#1#2#3{%
    \NR@gettitle{#3}%
    \NRorg@opargbegintheorem{#1}{#2}{#3}%
  }%
}
\AtBeginDocument{%
  \@ifpackageloaded{amsthm}{%
    \let\NRorg@begintheorem\@begintheorem
    \begingroup
      \edef\x{macro:\string#1\string#2[\string#3]}%
      \@onelevel@sanitize\x
      \def\y#1->#2\@nil{#1}%
      \edef\z{\expandafter\y\meaning\@begintheorem->\@nil}%
      \@onelevel@sanitize\z
    \expandafter\endgroup
    \ifx\x\z
      \def\@begintheorem#1#2[#3]{%
        \NR@gettitle{#3}%
        \NRorg@begintheorem{#1}{#2}[{#3}]%
      }%
    \fi
  }{}%
}
%    \end{macrocode}
%
% \subsection{Referencing stuff}
%
% We default the label and hypertext reference to be empty.
%    \begin{macrocode}
\providecommand*{\@currentlabelname}{}
\providecommand*{\@currentHref}{}
%    \end{macrocode}
% Since the second part of the preserved reference now has five parts
% (number, page, name, hypertext reference and file), we need extra utility
% macros:
%    \begin{macro}{\@firstoffive}
%    \begin{macrocode}
\long\def\@firstoffive#1#2#3#4#5{#1}
%    \end{macrocode}
%    \end{macro}
%    \begin{macro}{\@secondoffive}
%    \begin{macrocode}
\long\def\@secondoffive#1#2#3#4#5{#2}
%    \end{macrocode}
%    \end{macro}
%    \begin{macro}{\@thirdoffive}
%    \begin{macrocode}
\long\def\@thirdoffive#1#2#3#4#5{#3}
%    \end{macrocode}
%    \end{macro}
%    \begin{macro}{\@fourthoffive}
%    \begin{macrocode}
\long\def\@fourthoffive#1#2#3#4#5{#4}
%    \end{macrocode}
%    \end{macro}
%    \begin{macro}{\@fifthoffive}
%    \begin{macrocode}
\long\def\@fifthoffive#1#2#3#4#5{#5}
%    \end{macrocode}
%    \end{macro}
% We have to redefine |\ref|, |\Ref| and |\pageref| to know about extra
% reference elements. Make them robust, and compatible with Babel.
% We also define starred versions so that it doesn't depend on hyperref
% if they are defined or not. We use the same internal names as hyperref.
% The Babel switch is mimicked if not already defined.
% Some support for package \xpackage{showkeys} is implemented. If
% option |notref| of that package is set, |\SK@ref| has the
% meaning of |\@empty|.
%    \begin{macro}{\@safe@activestrue}
%    \begin{macrocode}
\providecommand*\@safe@activestrue{}%
%    \end{macrocode}
%    \end{macro}
%    \begin{macro}{\@safe@activesfalse}
%    \begin{macrocode}
\providecommand*\@safe@activesfalse{}%
%    \end{macrocode}
%    \end{macro}
%    \begin{macro}{\NR@ref@showkeys}
%    \begin{macrocode}
\newcommand\NR@ref@showkeys[1]{}%
%    \end{macrocode}
%    \end{macro}
%    \begin{macro}{\NR@setref}
%    \begin{macrocode}
\def\NR@setref#1{%
  \begingroup
    \@safe@activestrue
    \expandafter
  \endgroup
  \expandafter\NR@@setref\csname r@#1\endcsname
}
%    \end{macrocode}
%    \end{macro}
%    \begin{macro}{\NR@@setref}
%    \begin{macrocode}
\def\NR@@setref#1{%
  \ifx\@undefined#1%
    \let#1\relax
  \fi
  \@setref#1%
}
%    \end{macrocode}
%    \end{macro}
%    \begin{macro}{\T@ref}
%    \begin{macrocode}
\def\T@ref#1{%
  \NR@ref@showkeys{#1}%
  \NR@setref{#1}\@firstoffive{#1}%
}%
%    \end{macrocode}
%    \end{macro}
%    \begin{macro}{\@refstar}
%    \begin{macrocode}
\def\@refstar#1{%
  \NR@ref@showkeys{#1}%
  \NR@setref{#1}\@firstoffive{#1}%
}%
%    \end{macrocode}
%    \end{macro}
%    \begin{macro}{\T@pageref}
%    \begin{macrocode}
\def\T@pageref#1{%
  \NR@ref@showkeys{#1}%
  \NR@setref{#1}\@secondoffive{#1}%
}%
%    \end{macrocode}
%    \end{macro}
%    \begin{macro}{\@pagerefstar}
%    \begin{macrocode}
\def\@pagerefstar#1{%
  \NR@ref@showkeys{#1}%
  \NR@setref{#1}\@secondoffive{#1}%
}%
%    \end{macrocode}
%    \end{macro}%
%    \begin{macro}{\T@Ref}
%    \begin{macrocode}
\def\T@Ref#1{%
    \NR@ref@showkeys{#1}%
    \NR@setref{#1}\NR@MakeUppercaseFirstOfFive{#1}%
  }%
\def\NR@MakeUppercaseFirstOfFive#1#2#3#4#5{%
    \MakeUppercase#1%
  }%
%    \end{macrocode}
%    \end{macro}
%    \begin{macro}{\@Refstar}
%    \begin{macrocode}
\def\@Refstar#1{%
    \NR@ref@showkeys{#1}%
    \NR@setref{#1}\NR@MakeUppercaseFirstOfFive{#1}%
  }%
%    \end{macrocode}
%    \end{macro}%
%    \begin{macrocode}
\ifNR@showkeys
  \ifx\SK@ref\@empty
  \else
  \renewcommand\NR@ref@showkeys[1]{%
    \@safe@activestrue
    \SK@\SK@@ref{#1}%
    \@safe@activesfalse}
  \fi
\fi
%    \end{macrocode}
%    Package \xpackage{hyperref} uses the same
%    definitions of |\ref|, |\Ref| and
%    |\pageref| but has to redefine the non-starred command to
%    add the hyperlink support.
%    |\DeclareRobustCommand| is used by \xpackage{hyperref}.
%    Package |nameref| can be loaded before, after
%    \xpackage{hyperref} or after
%    \xpackage{hyperref}'s definitions in |\AtBeginDocument|.
%    We have to postpone the definition to |\AtBeginDocument| to
%    avoid that showkeys overwrites them again.
%    \begin{macrocode}
\AddToHook{begindocument}{%
%    \end{macrocode}
%    \begin{macro}{\ref}
%    \begin{macrocode}
  \DeclareDocumentCommand\ref{s}
   {\IfBooleanTF{#1}{\@refstar}{\T@ref}}%
%    \end{macrocode}
%    \end{macro}
%    \begin{macro}{\pageref}
%    \begin{macrocode}
  \DeclareDocumentCommand\pageref{s}
   {\IfBooleanTF{#1}{\@pagerefstar}{\T@pageref}}%
%    \end{macrocode}
%    \end{macro}
%    \begin{macro}{\Ref}
%    \begin{macrocode}
  \DeclareDocumentCommand\Ref{s}
   {\IfBooleanTF{#1}{\@Refstar}{\T@Ref}}%
%    \end{macrocode}
%    \end{macro}
%    \begin{macrocode}
}
%    \end{macrocode}
%    \begin{macrocode}
\DeclareHookRule{begindocument}{showkeys}{before}{nameref}
%    \end{macrocode}
%
%
% \subsection{Usage and frontend}
% Access the (third) name part with |\nameref|;
%    \begin{macro}{\T@nameref}
%    \begin{macrocode}
\def\T@nameref#1{%
  \NR@ref@showkeys{#1}%
  \begingroup
    \let\label\@gobble
    \NR@setref{#1}\@thirdoffive{#1}%
  \endgroup
}%
%    \end{macrocode}
%    \end{macro}
%    \begin{macro}{\@namerefstar}
%    \begin{macrocode}
\def\@namerefstar#1{%
  \NR@ref@showkeys{#1}%
  \begingroup
    \let\label\@gobble
    \NR@setref{#1}\@thirdoffive{#1}%
  \endgroup
}%
%    \end{macrocode}
%    \end{macro}
%    \begin{macrocode}
%    \end{macrocode}
%    \begin{macro}{\nameref}
%    \begin{macrocode}
\@ifundefined{nameref}
  {
   \NewDocumentCommand\nameref{s}
    {\IfBooleanTF{#1}{\@namerefstar}{\T@nameref}}
  }
  {
    \PackageWarning{nameref}{Redefining \string\nameref}
    \RenewDocumentCommand\nameref{s}
    {\IfBooleanTF{#1}{\@namerefstar}{\T@nameref}}
  }
%    \end{macrocode}
%    \end{macro}
%    \begin{macro}{\Nameref}
%    An extended form which gives title and page number.
%    Not really useful, quotes are english, and no starred version.
%    varioref can do that better. Remove??
%    \begin{macrocode}
\def\Nameref#1{`\nameref{#1}' on page~\pageref{#1}}
%    \end{macrocode}
%    \end{macro}
%
%    \begin{macro}{\Sectionformat}
%    The default for |\Sectionformat|
%    \begin{macrocode}
\providecommand\Sectionformat[2]{#1}
%    \end{macrocode}
%    \end{macro}
%
% \subsection{Package ifthen support}
%
%    Package ifthen's \cmd{\isodd} does not work with
%    \cmd{\pageref} because of the extra link and the
%    star form. Therefore we need an expandable variant.
%
%    Unfortunately there is no hook in \cmd{\ifthenelse} in order
%    to provide expandable versions without link of \cmd{\ref}
%    and \cmd{\pageref}. As possible target I could only find
%    \cmd{\begingroup}:
%    Changed/disabled 2022-04-13: ifthen takes care itself to
%    change \cmd{\pageref} to an expandable version.
%    \begin{macrocode}
\@ifundefined{@kernel@pageref@exp}
   {\@ifpackageloaded{ifthen}{%
     \let\NROrg@ifthenelse\ifthenelse
     \def\ifthenelse{%
       \let\begingroup\NR@TE@begingroup
       \NROrg@ifthenelse
     }%
     \let\NROrg@begingroup\begingroup
     \def\NR@TE@begingroup{%
       \let\begingroup\NROrg@begingroup
       \begingroup
       \def\ref##1{\getrefbykeydefault{##1}{}{??}}%
       \def\pageref##1{\getrefbykeydefault{##1}{page}{0}}%
     }%
   }{}}{}
%    \end{macrocode}
%
% \subsection{Compatibility with varioref}
%
% [This section was derived by Corey Minyard \texttt{minyard@acm.org}
% from the \xpackage{varioref} package]
% Changed in 2019-09-12 by Frank Mittelbach to adapt to changes in
% varioref
%    \begin{macrocode}
\AtBeginDocument{%
  \ifx\csname @@vpageref\endcsname\relax
  \else
%    \end{macrocode}
%    This requires a varioref newer than 2019-09-12
%    \begin{macrocode}
   \let\NR@@vpageref\@@vpageref
   \def\@@vpageref#1[#2]#3{%
      \begingroup
        \csname @safe@activestrue\endcsname
        \edef\x{#3}%
        \@onelevel@sanitize\x
        \toks@{%
          \NR@@vpageref{#1}[{#2}]%
        }%
        \edef\x{\endgroup
          \the\toks@{\x}%
        }%
      \x
    }%
    \DeclareRobustCommand\vnameref[1]{%
      \unskip~\nameref{#1} %
      \@vpageref[\unskip]{#1}%
    }%
  \fi
}
%    \end{macrocode}
%
% \subsection{Support for package \xpackage{listings}}
%
%    \begin{macrocode}
\AtBeginDocument{%
  \@ifpackageloaded{listings}{%
    \@ifundefined{NR@nopatch@listings}
     {%
       \let\NROrg@lst@MakeCaption\lst@MakeCaption
       \def\lst@MakeCaption{%
         \ifx\lst@label\@empty
         \else
           \let\@currentlabelname\lst@@caption
         \fi
         \NROrg@lst@MakeCaption
       }%
      }{}%
  }{}%
}
%    \end{macrocode}
%
%    \begin{macrocode}
%</package>
%    \end{macrocode}
%
% \Finale
\endinput
