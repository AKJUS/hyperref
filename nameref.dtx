% \iffalse
%% File: nameref.dtx
%% Copyright (C) 1995-1999 Sebastian Rahtz
%%               2000 Sebastian Rahtz, Heiko Oberdiek
%%               2001-2012 Heiko Oberdiek
%%               2016-2019 Oberdiek Package Suport Group
%%               2019-2022 The LaTeX Project
%%      https://github.com/latex3/hyperref/issues
%%
%% This file is part of the `Hyperref Bundle'.
%% -------------------------------------------
%%
%% This work may be distributed and/or modified under the
%% conditions of the LaTeX Project Public License, either version 1.3
%% of this license or (at your option) any later version.
%% The latest version of this license is in
%%   http://www.latex-project.org/lppl.txt
%% and version 1.3 or later is part of all distributions of LaTeX
%% version 2005/12/01 or later.
%%
%% This work has the LPPL maintenance status `maintained'.
%%
%% The Current Maintainer of this work is The LaTeX Project.
%%
%% The list of all files belonging to the `Hyperref Bundle' is
%% given in the file `manifest.txt'.
%
%<*dtx>
\ProvidesFile{nameref.dtx}
%</dtx>
%<package>\NeedsTeXFormat{LaTeX2e}
%<package>\ProvidesPackage{nameref}
%<driver>\ProvidesFile{nameref.drv}
% \fi
% \ProvidesFile{nameref.dtx}
  [2021-04-02 v2.47 Cross-referencing by name of section]%
%
%
% \iffalse
%<*driver>
\PassOptionsToPackage{numbered}{hypdoc}
\documentclass{ltxdoc}
\usepackage[numbered]{hypdoc}
\usepackage{bmhydoc}
\usepackage{hologo}
\EnableCrossrefs
\CodelineIndex
\newcommand*{\xpackage}[1]{\textsf{#1}}
\begin{document}
  \GetFileInfo{nameref.sty}%
  \title{%
    Section name references in \LaTeX
    \thanks{%
      This file has version number \fileversion, %
      last revised \filedate.%
    }%
  }%
  \author{Sebastian Rahtz (deceased)\\%
    Heiko Oberdiek (inactive)\\%
    The \LaTeX\ Project\\%
    \texttt{https://github.com/latex3/hyperref/issues}}%
  \date{\filedate}%
  \maketitle
  \makeatletter
  \@ifundefined{HyperrefOverview}{}{%
    \HyperrefOverview*%
  }%
  \tableofcontents
  \DocInput{nameref.dtx}%
  \PrintIndex
\end{document}
%</driver>
% \fi
%
% \StopEventually{}
%
% \section{Introduction}
% Cross-referencing to include the \emph{name} of the section, rather than
% just the number or page. This works by redefining some of the inside
% \LaTeX\ macros, so it is vulnerable to future changes.
% \section{History}
% Sebastian Rahtz, for Lou Burnard, March 15th 1994;\\
% SPQR CERN July 1994;\\
% Last mod. Sept. 3th MG;\\
% Sept. 19th SPQR;\\
% April 11th 1995 SPQR (added section formatting hook);\\
% April 14th 1995 SPQR (compatibility with hyperref);\\
% June 22 1995 SPQR (removed typeout from Sectionformat);\\
% September 22 1997 added varioref hacks by Corey Minyard;\\
% September 28th changes by David Carlisle.\\
% More recent changes are recorded in ChangeLog.
%
% The versions before 2009/11/27 v2.32 did not expand the
% title strings (of \cs{section}, \cs{caption}, \dots).
% Then the behaviour switched to expanding, because this
% allows to remove unwanted macros (\cs{label}, \dots).
% Despite this method is used in package \textsf{titleref}
% and class \textsf{memoir}, the first feedbacks let me conclude
% that this is not robust enough. Therefore I wrote package
% \textsf{gettitlestring} that uses the non-expanding
% method as default. It also allows me to use this package
% in package \textsf{zref-titleref}.
% See the package documentation of package \textsf{gettitlestring}
% how to change the method. It also provides a hook to remove
% additional unwanted macros, not supported by the package.
% Example:
%\begin{quote}
%\begin{verbatim}
%\usepackage{gettitlestring}
%\GetTitleStringDisableCommands{%
%  \let\mylabel\@gobble
%  \renewcommand\myindex[2]{}%
%}
%\end{verbatim}
%\end{quote}
%
% \section{Implementation}
%    \begin{macrocode}
%<*package>
%    \end{macrocode}
%
%    \begin{macrocode}
\RequirePackage{refcount}[2006/02/12]
\RequirePackage{gettitlestring}[2009/12/18]
\RequirePackage{ltxcmds}[2009/12/12]
%    \end{macrocode}
%
% We redefine |\label| so that it also writes the name of the
% current section to the .aux file; if the name ends in a dot,
% we zap it. To allow for the \xpackage{hyperref} package, also write
% fourth and fifth fields (empty by default;
% used for cross-ref name, and file).
%    \begin{macro}{\NR@strip@period}
%    \begin{macrocode}
\def\NR@strip@period#1.\ltx@empty#2\@nil{#1}
%    \end{macrocode}
%    \end{macro}
%
%    \begin{macro}{\strip@period}
% Macro \cs{strip@period} was removed in \xpackage{nameref}
% 2012/07/28 v2.41 and reinserted in 2012/07/31 v2.42 for compatibilty
% reasons with package \xpackage{subfig}.
%    \begin{macrocode}
\def\strip@period#1.\relax#2\@@@{#1}
%    \end{macrocode}
%    \end{macro}
%
%    \begin{macrocode}
\def\NR@sanitize@labelname{%
  \@onelevel@sanitize\@currentlabelname
  \edef\@currentlabelname{%
    \expandafter\NR@strip@period\@currentlabelname
    \ltx@empty.\ltx@empty\@nil
  }%
}
%    \end{macrocode}
%
% \subsection{\cs{label} stuff}
%
%    Here is tested, whether package \xpackage{showkeys} is loaded.
%    But if option final is used, the macros of \xpackage{showkeys}
%    are not defined, so we have to check an additional
%    test.
%    \begin{macro}{\ifNR@showkeys}
%    \begin{macrocode}
\newif\ifNR@showkeys
\NR@showkeysfalse
\@ifpackageloaded{showkeys}{%
  \begingroup
    \@ifundefined{SK@@label}{%
    }{%
      \global\NR@showkeystrue
    }%
  \endgroup
}{}
%    \end{macrocode}
%    \end{macro}
%    \begin{macro}{\label@hook}
%    \begin{macrocode}
\providecommand*{\label@hook}{}
\ltx@GlobalAppendToMacro{\label@hook}{%
  \NR@sanitize@labelname
}
%    \end{macrocode}
%    \end{macro}
%    \begin{macro}{\label}
%    \begin{macrocode}
\def\label#1{%
  \@bsphack
  \begingroup
    \def\label@name{#1}%
    \label@hook
    \protected@write\@auxout{}{%
      \string\newlabel{#1}{%
        {\@currentlabel}%
        {\thepage}%
        {\@currentlabelname}%
        {\@currentHref}{}%
      }%
    }%
  \endgroup
  \@esphack
}%
\ifNR@showkeys
  \def\label#1{%
    \@bsphack
    \SK@\SK@@label{#1}%
    \begingroup
      \def\label@name{#1}%
      \label@hook
      \protected@write\@auxout{}{%
        \string\newlabel{#1}{%
          {\@currentlabel}%
          {\thepage}%
          {\@currentlabelname}%
          {\@currentHref}{}%
        }%
      }%
    \endgroup
    \@esphack
  }%
\fi
%    \end{macrocode}
%    \end{macro}
%    \begin{macro}{\ltx@label}
%    \begin{macrocode}
\let\ltx@label\label
%    \end{macrocode}
%    \end{macro}
% Needed for the \emph{subeqnarray} package.
%    \begin{macro}{\slabel}
%    \begin{macrocode}
\@ifundefined{slabel}{}{%
  \def\slabel#1{%
    \@bsphack
    \if@filesw
      \begingroup
        \def\label@name{#1}%
        \label@hook
        \let\thepage\relax
        \def\protect{\noexpand\noexpand\noexpand}%
        \edef\@tempa{%
          \write\@auxout{%
            \string\newlabel{#1}{%
              {\thesubequation}%
              {\thepage}%
              {\@currentlabelname}%
              {\@currentHref}{}%
            }%
          }%
        }%
      \expandafter\endgroup
      \@tempa
      \if@nobreak\ifvmode\nobreak\fi\fi
    \fi
    \@esphack
  }%
}
%    \end{macrocode}
%    \end{macro}
% Overload an AMS \hologo{LaTeX} command, which uses |\newlabel|. Sigh!
%    \begin{macro}{\@writetocindents@}
%    \begin{macrocode}
\def\@writetocindents@{%
  \begingroup
  \@for\@tempa:=-1,0,1,2,3\do{%
    \immediate\write\@auxout{%
      \string\newlabel{tocindent\@tempa}{%
        \csname r@tocindent\@tempa\endcsname{}{}{}{}%
      }%
    }%
  }%
  \endgroup
}
%    \end{macrocode}
%    \end{macro}
%
% \subsection{Changes of section macros}
%
% Add to the underlying section heading macros so that they
% note the section name for use by label.
%
% If a section heading or the like has a |\label| in it,
% we need to extract it, or subsequent processing breaks.
%
%    \begin{macro}{\NR@gettitle}
%    \begin{macrocode}
\def\NR@gettitle#1{%
  \GetTitleString{#1}%
  \let\@currentlabelname\GetTitleStringResult
}
%    \end{macrocode}
%    \end{macro}
%
% First the numbered sections. While we are about it, put in
% a useful section formatting macro.
%    \begin{macro}{\NR@sect}
%    \begin{macrocode}
\let\NR@sect\@sect
%    \end{macrocode}
%    \end{macro}
%    \begin{macrocode}
\newcounter{section@level}
%    \end{macrocode}
%    \begin{macro}{\@sect}
%    \begin{macrocode}
\def\@sect#1#2#3#4#5#6[#7]#8{%
  \setcounter{section@level}{#2}%
  \NR@gettitle{#7}%
  \NR@sect{#1}{#2}{#3}{#4}{#5}{#6}[{#7}]{\Sectionformat{#8}{#2}}%
}
%    \end{macrocode}
%    \end{macro}
% and now the unnumbered ones
%    \begin{macro}{\NR@ssect}
%    \begin{macrocode}
\let\NR@ssect\@ssect
%    \end{macrocode}
%    \end{macro}
%    \begin{macro}{\NR@ssect}
%    \begin{macrocode}
\def\@ssect#1#2#3#4#5{%
  \NR@gettitle{#5}%
  \NR@ssect{#1}{#2}{#3}{#4}{\Sectionformat{#5}{#1}}%
}
%    \end{macrocode}
%    \end{macro}
%
% Support for class memoir.
%
%    \begin{macrocode}
\ltx@IfUndefined{ifheadnameref}{%
}{%
  \ltx@IfUndefined{M@sect}{%
  }{%
    \let\NRorg@M@sect\M@sect
    \def\M@sect#1#2#3#4#5#6[#7][#8]{%
      \ifheadnameref
        \NR@gettitle{#8}%
      \else
        \NR@gettitle{#7}%
      \fi
      \NRorg@M@sect{#1}{#2}{#3}{#4}{#5}{#6}[{#7}][{#8}]%
    }%
  }%
}
%    \end{macrocode}
%
% Parts and chapters are treated differently. sigh.
% \cs{@part} and \cs{@spart} are \cs{long} because of class memoir.
%
%    \begin{macrocode}
\let\NR@part\@part
\long\def\@part[#1]#2{%
  \NR@gettitle{#1}%
  \NR@part[{#1}]{#2}%
}
\let\NR@spart\@spart
\long\def\@spart#1{%
  \NR@gettitle{#1}%
  \NR@spart{#1}%
}
\let\NR@chapter\@chapter
\def\@chapter[#1]#2{%
  \NR@gettitle{#1}%
  \NR@chapter[{#1}]{#2}%
}
\@ifclassloaded{memoir}{%
  \ltx@IfUndefined{ifheadnameref}{%
  }{%
    \def\@chapter[#1]#2{%
      \ltx@IfUndefined{ch@pt@c}{%
        \NR@gettitle{#1}%
      }{%
        \ifx\ch@pt@c\ltx@empty
          \NR@gettitle{#2}%
        \else
          \def\NR@temp{#1}%
          \ifx\NR@temp\ltx@empty
            \expandafter\NR@gettitle\expandafter{\ch@pt@c}%
          \else
            \ifheadnameref
              \NR@gettitle{#1}%
            \else
              \expandafter\NR@gettitle\expandafter{\ch@pt@c}%
            \fi
          \fi
        \fi
      }%
      \NR@chapter[{#1}]{#2}%
    }%
  }%
}{}
\let\NR@schapter\@schapter
\def\@schapter#1{%
  \NR@gettitle{#1}%
  \NR@schapter{#1}%
}
%    \end{macrocode}
%
% Captions
%
%    \begin{macrocode}
\AtBeginDocument{%
  \@ifpackageloaded{hyperref}{%
    \@ifpackagelater{hyperref}{2009/12/05}\@gobble\@firstofone
  }\@firstofone
  {%
    \let\NR@@caption\@caption
    \long\def\@caption#1[#2]{%
      \NR@gettitle{#2}%
      \NR@@caption{#1}[{#2}]%
    }%
  }%
%    \end{macrocode}
%
%    \begin{macrocode}
  \@ifpackageloaded{longtable}{%
    \let\NRorg@LT@c@ption\LT@c@ption
    \def\LT@c@ption#1[#2]#3{%
      \NRorg@LT@c@ption{#1}[{#2}]{#3}%
      \def\@tempa{#2}%
      \ifx\@tempa\@empty
      \else
        \NR@gettitle{#2}%
      \fi
    }%
  }{}%
%    \end{macrocode}
%
% Environment `description'.
%    \begin{macrocode}
  \ltx@IfUndefined{descriptionlabel}{%
  }{%
    \let\NRorg@descriptionlabel\descriptionlabel
    \def\descriptionlabel#1{%
      \NR@gettitle{#1}%
      \NRorg@descriptionlabel{#1}%
    }%
  }%
}
%    \end{macrocode}
%
% Compatibility for package titlesec.
%    \begin{macrocode}
\AtBeginDocument{%
  \@ifpackageloaded{titlesec}{%
    \let\NRorg@ttl@sect@i\ttl@sect@i
    \def\ttl@sect@i#1#2[#3]#4{%
      \NR@gettitle{#4}%
      \NRorg@ttl@sect@i{#1}{#2}[{#3}]{#4}%
    }%
  }{}%
}
%    \end{macrocode}
%
% \subsubsection{Class beamer}
%
%    \begin{macrocode}
\@ifclassloaded{beamer}{%
  \def\NR@PatchBeamer#1{%
    \expandafter\let\csname NRorg@beamer@#1\expandafter\endcsname
                    \csname beamer@#1\endcsname
    \begingroup
      \edef\x{\endgroup
        \long\def\expandafter\noexpand\csname beamer@#1\endcsname
            [####1]{% hash-ok
          \noexpand\NR@gettitle{####1}%
          \expandafter\noexpand\csname NRorg@beamer@#1\endcsname
          [{####1}]%
        }%
      }%
    \x
  }%
  \NR@PatchBeamer{section}%
  \NR@PatchBeamer{subsection}%
  \NR@PatchBeamer{subsubsection}%
%    \end{macrocode}
%
%    Theorem names.
%    \begin{macrocode}
  \addtobeamertemplate{theorem begin}{%
    \expandafter\NR@gettitle\expandafter{\inserttheoremaddition}%
  }{}%
%    \end{macrocode}
%    \begin{macrocode}
  \@ifdefinable{\NRorg@beamer@descriptionitem}{%
    \let\NRorg@beamer@descriptionitem\beamer@descriptionitem
    \renewcommand{\beamer@descriptionitem}[1]{%
      \NR@gettitle{#1}%
      \NRorg@beamer@descriptionitem{#1}%
    }%
  }%
}{}
%    \end{macrocode}
%
% \subsubsection{Theorems}
%
%    \begin{macrocode}
\ltx@IfUndefined{@opargbegintheorem}{}{%
  \let\NRorg@opargbegintheorem\@opargbegintheorem
  \def\@opargbegintheorem#1#2#3{%
    \NR@gettitle{#3}%
    \NRorg@opargbegintheorem{#1}{#2}{#3}%
  }%
}
\AtBeginDocument{%
  \@ifpackageloaded{amsthm}{%
    \let\NRorg@begintheorem\@begintheorem
    \begingroup
      \edef\x{macro:\string#1\string#2[\string#3]}%
      \@onelevel@sanitize\x
      \def\y#1->#2\@nil{#1}%
      \edef\z{\expandafter\y\meaning\@begintheorem->\@nil}%
      \@onelevel@sanitize\z
    \expandafter\endgroup
    \ifx\x\z
      \def\@begintheorem#1#2[#3]{%
        \NR@gettitle{#3}%
        \NRorg@begintheorem{#1}{#2}[{#3}]%
      }%
    \fi
  }{}%
}
%    \end{macrocode}
%
% \subsection{Referencing stuff}
%
% We default the label and hypertext reference to be empty.
%    \begin{macrocode}
\providecommand*{\@currentlabelname}{}
\providecommand*{\@currentHref}{}
%    \end{macrocode}
% Since the second part of the preserved reference now has five parts
% (number, page, name, hypertext reference and file), we need extra utility
% macros:
%    \begin{macro}{\@firstoffive}
%    \begin{macrocode}
\long\def\@firstoffive#1#2#3#4#5{#1}
%    \end{macrocode}
%    \end{macro}
%    \begin{macro}{\@secondoffive}
%    \begin{macrocode}
\long\def\@secondoffive#1#2#3#4#5{#2}
%    \end{macrocode}
%    \end{macro}
%    \begin{macro}{\@thirdoffive}
%    \begin{macrocode}
\long\def\@thirdoffive#1#2#3#4#5{#3}
%    \end{macrocode}
%    \end{macro}
%    \begin{macro}{\@fourthoffive}
%    \begin{macrocode}
\long\def\@fourthoffive#1#2#3#4#5{#4}
%    \end{macrocode}
%    \end{macro}
%    \begin{macro}{\@offive}
%    \begin{macrocode}
\long\def\@fifthoffive#1#2#3#4#5{#5}
%    \end{macrocode}
%    \end{macro}
% We have to redefine |\ref| and |\pageref| to know about extra
% reference elements. Make them robust, and compatible with Babel.
% The Babel switch is mimicked if not already defined.
% Some support for package \xpackage{showkeys} is implemented. If
% option |notref| of that package is set, |\SK@ref| has the
% meaning of |\@empty|.
%    \begin{macro}{\@safe@activestrue}
%    \begin{macrocode}
\providecommand*\@safe@activestrue{}%
%    \end{macrocode}
%    \end{macro}
%    \begin{macro}{\@safe@activesfalse}
%    \begin{macrocode}
\providecommand*\@safe@activesfalse{}%
%    \end{macrocode}
%    \end{macro}
%    \begin{macro}{\NR@setref}
%    \begin{macrocode}
\def\NR@setref#1{%
  \begingroup
    \@safe@activestrue
    \expandafter
  \endgroup
  \expandafter\NR@@setref\csname r@#1\endcsname
}
%    \end{macrocode}
%    \end{macro}
%    \begin{macro}{\NR@@setref}
%    \begin{macrocode}
\def\NR@@setref#1{%
  \ifx\@undefined#1%
    \let#1\relax
  \fi
  \@setref#1%
}
%    \end{macrocode}
%    \end{macro}
%    \begin{macro}{\T@ref}
%    \begin{macrocode}
\def\T@ref#1{%
  \NR@setref{#1}\@firstoffive{#1}%
}%
%    \end{macrocode}
%    \end{macro}
%    \begin{macro}{\T@pageref}
%    \begin{macrocode}
\def\T@pageref#1{%
  \NR@setref{#1}\@secondoffive{#1}%
}%
%    \end{macrocode}
%    \end{macro}
%    \begin{macrocode}
\ifNR@showkeys
  \ifx\SK@ref\@empty
  \else
%    \end{macrocode}
%    \begin{macro}{\T@ref}
%    \begin{macrocode}
    \def\T@ref#1{%
      \@safe@activestrue
      \SK@\SK@@ref{#1}%
      \@safe@activesfalse
      \NR@setref{#1}\@firstoffive{#1}%
    }%
%    \end{macrocode}
%    \end{macro}
%    \begin{macro}{\T@pageref}
%    \begin{macrocode}
    \def\T@pageref#1{%
      \@safe@activestrue
      \SK@\SK@@ref{#1}%
      \@safe@activesfalse
      \NR@setref{#1}\@secondoffive{#1}%
    }%
%    \end{macrocode}
%    \end{macro}
%    \begin{macrocode}
  \fi
\fi
%    \end{macrocode}
%    Package \xpackage{hyperref} uses its own definitions of |\ref| and
%    |\pageref| because it extends the syntax by the star form.
%    |\DeclareRobustCommand| is used by \xpackage{hyperref}.
%    Package |nameref| can be loaded before, after
%    \xpackage{hyperref} or after
%    \xpackage{hyperref}'s definitions in |\AtBeginDocument|.
%    \begin{macrocode}
\@ifpackageloaded{hyperref}{%
}{%
%    \end{macrocode}
%    \begin{macro}{\ref}
%    \begin{macrocode}
  \DeclareRobustCommand{\ref}{\T@ref}%
%    \end{macrocode}
%    \end{macro}
%    \begin{macro}{\pageref}
%    \begin{macrocode}
  \DeclareRobustCommand{\pageref}{\T@pageref}%
%    \end{macrocode}
%    \end{macro}
%    \begin{macro}{\Ref}
%    \begin{macrocode}
   \def\T@Ref#1{%
       \NR@setref{#1}\NR@MakeUppercaseFirstOfFive{#1}%
     }%
   \def\NR@MakeUppercaseFirstOfFive#1#2#3#4#5{%
       \MakeUppercase#1%
     }%
   \DeclareRobustCommand*{\Ref}{%
     \@ifstar\T@Ref\T@Ref}%
}
%    \end{macrocode}
%    \end{macro}
%
% \subsection{Usage and frontend}
% Access the (third) name part with |\nameref|;
%    \begin{macro}{\T@nameref}
%    \begin{macrocode}
\def\T@nameref#1{%
  \begingroup
    \let\label\@gobble
    \NR@setref{#1}\@thirdoffive{#1}%
  \endgroup
}%
\ifNR@showkeys
  \ifx\SK@ref\@empty
  \else
    \def\T@nameref#1{%
      \@safe@activestrue
      \SK@\SK@@ref{#1}%
      \@safe@activesfalse
      \begingroup
        \let\label\@gobble
        \NR@setref{#1}\@thirdoffive{#1}%
      \endgroup
    }%
  \fi
\fi
%    \end{macrocode}
%    \end{macro}
%    \begin{macrocode}
%    \end{macrocode}
%    \begin{macro}{\nameref}
%    \begin{macrocode}
  \DeclareRobustCommand*{\nameref}{%
    \@ifstar\T@nameref\T@nameref
  }%
%    \end{macrocode}
%    \end{macro}
%    \begin{macro}{\Nameref}
%    An extended form which gives title and page number.
%    \begin{macrocode}
\def\Nameref#1{`\nameref{#1}' on page~\pageref{#1}}
%    \end{macrocode}
%    \end{macro}
%
%    \begin{macro}{\Sectionformat}
%    The default for |\Sectionformat|
%    \begin{macrocode}
\providecommand\Sectionformat[2]{#1}
%    \end{macrocode}
%    \end{macro}
%
% \subsection{Package ifthen support}
%
%    Package ifthen's \cmd{\isodd} does not work with
%    \cmd{\pageref} because of the extra link and the
%    star form. Therefore we need an expandable variant.
%
%    Unfortunately there is no hook in \cmd{\ifthenelse} in order
%    to provide expandable versions without link of \cmd{\ref}
%    and \cmd{\pageref}. As possible target I could only find
%    \cmd{\begingroup}:
%    \begin{macrocode}
\@ifpackageloaded{ifthen}{%
  \let\NROrg@ifthenelse\ifthenelse
  \def\ifthenelse{%
    \let\begingroup\NR@TE@begingroup
    \NROrg@ifthenelse
  }%
  \let\NROrg@begingroup\begingroup
  \def\NR@TE@begingroup{%
    \let\begingroup\NROrg@begingroup
    \begingroup
    \def\ref##1{\getrefbykeydefault{##1}{}{??}}%
    \def\pageref##1{\getrefbykeydefault{##1}{page}{0}}%
  }%
}{}
%    \end{macrocode}
%
% \subsection{Compatibility with varioref}
%
% [This section was derived by Corey Minyard \texttt{minyard@acm.org}
% from the \xpackage{varioref} package]
% Changed in 2019-09-12 by Frank Mittelbach to adapt to changes in
% varioref
%    \begin{macrocode}
\AtBeginDocument{%
  \ifx\csname @@vpageref\endcsname\relax
  \else
%    \end{macrocode}
%    This requires a varioref newer than 2019-09-12
%    \begin{macrocode}
   \let\NR@@vpageref\@@vpageref
   \def\@@vpageref#1[#2]#3{%
      \begingroup
        \csname @safe@activestrue\endcsname
        \edef\x{#3}%
        \@onelevel@sanitize\x
        \toks@{%
          \NR@@vpageref{#1}[{#2}]%
        }%
        \edef\x{\endgroup
          \the\toks@{\x}%
        }%
      \x
    }%
    \DeclareRobustCommand\vnameref[1]{%
      \unskip~\nameref{#1} %
      \@vpageref[\unskip]{#1}%
    }%
  \fi
}
%    \end{macrocode}
%
% \subsection{Support for package \xpackage{listings}}
%
%    \begin{macrocode}
\AtBeginDocument{%
  \@ifpackageloaded{listings}{%
    \let\NROrg@lst@MakeCaption\lst@MakeCaption
    \def\lst@MakeCaption{%
      \ifx\lst@label\@empty
      \else
        \let\@currentlabelname\lst@@caption
      \fi
      \NROrg@lst@MakeCaption
    }%
  }{}%
}
%    \end{macrocode}
%
%    \begin{macrocode}
%</package>
%    \end{macrocode}
%
% \Finale
\endinput
