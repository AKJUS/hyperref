% \iffalse meta-comment
%
% Copyright (C) 1993-2024 David Carlisle
%
% This file is part of the `Hyperref Bundle'.
% -------------------------------------------
%
% This work may be distributed and/or modified under the
% conditions of the LaTeX Project Public License, either version 1.3
% of this license or (at your option) any later version.
% The latest version of this license is in
%   http://www.latex-project.org/lppl.txt
% and version 1.3 or later is part of all distributions of LaTeX
% version 2005/12/01 or later.
%
% This work has the LPPL maintenance status `maintained'.
%
% The Current Maintainer of this work is the LaTeX Project.
%
% The list of all files belonging to the `Hyperref Bundle' is
% given in the file `manifest.txt'.
%
% \fi
% \iffalse
%% File: xr-hyper.dtx Copyright (C) 1993-2024 David Carlisle
%
%<package>\NeedsTeXFormat{LaTeX2e}
%<package>\ProvidesPackage{xr-hyper}
%<package>         [2024-05-23 v7.01i eXternal References (DPC)]
%
%<*driver>
\documentclass{l3doc}
\usepackage{xr-hyper}
\GetFileInfo{xr-hyper.sty}
\begin{document}
\title{The \textsf{xr-hyper} package\thanks{This file
        has version number \fileversion, last
        revised \filedate.}}
\author{David Carlisle}
\date{\filedate}
\maketitle
\DocInput{xr-hyper.dtx}
\end{document}
%</driver>
% \fi
%
% \changes{v5.00}{1993/07/07}
%         {First DPC version (by agreement with J-PD).  New mechanism
%         (\cs{read} instead of \cs{input}). (xr)}
%
% \changes{v5.01}{1993/07/20}{Fix bug added by DPC, v5.00 did not import
%           aux files of \cs{include}'ed files. (Reported by J-PD) (xr)}
%
% \changes{v5.02}{1994/05/28}{Update for LaTeX2e}
% \changes{v5.03}{2018/10/01}{Fix for conditionals in aux file (xr)}
% \changes{v5.05}{2019/07/20}{include xcite (xr)}
%
% \changes{v6.00 beta2}{1997/10/06}
%         {}
%
% \changes{v6.00 beta3}{1999/09/01}
%         {Definitions of \cs{XR@addURL} corrected (Heiko Oberdiek)}
%
% \changes{v6.00 beta 4}{2000/03/22}
%         {\cmd\xdef changed to \cs{protected@xdef}} 
%
% \changes{v6.1}{2019/10/03}
%         {Incorporate updates to \pkg{xr} (including citation support)}
%
% \changes{v7.00e}{2020-05-14}
%         {add additional braces to \cs{r@}\meta{label} only if \pkg{hyperref} is not detected in the aux
%          of the source document (https://github.com/latex3/hyperref/issues/118)}
%
% \changes{v7.00z}{2023-05-15}
%         {unified and simplified the brace handling: all defined references have now
%          five brace groups. Together with the planed change in LaTeX 2023-06-01 this
%          will improve the combination of hyperref/non-hyperref documents.}
%          
% \changes{v7.01g}{2023-11-26}
%         {Always store the URL/file name of the external document to allow file links
%          even if it has no destinations.}
%  
% \changes{v7.01h}{2024-01-20}
%         {do not expand the labels when reading, (https://github.com/latex3/hyperref/issues/327)}
%         
% \changes{v7.01h}{2024-01-20}
%         {support \LaTeX\ properties.}
%        
% This package implements a system for eXternal References.
% 
% It is an extension of the \pkg{xr} package. It was developed to support the
% extended label syntax of the \pkg{hyperref} package and to enable active links
% to the external documents.
% 
% In the \LaTeX{} release 2023-06-01 the label syntax of \pkg{hyperref}
% and the \LaTeX{} kernel have been synchronized and there is no longer
% a need for two packages. \pkg{xr-hyper} already works with all documents --
% it is not required to load \pkg{hyperref} -- and its code will move in the
% next \LaTeX{} release into the \pkg{xr} package. Then \pkg{xr-hyper} can be
% deprecated.
% 
% \section{Usage}
% 
% \begin{syntax}
% \cs{externaldocument}\oarg{prefix}\texttt[nocite\texttt]\marg{document}\oarg{url}
% \end{syntax}
%
% If one document needs to refer to sections of another, say |aaa.tex|,
% then this package may be loaded in the main file, and the command
% \begin{verbatim}
% \externaldocument{aaa}
% \end{verbatim}
%  given in the preamble.
%
% Then you may use |\ref| and |\pageref| (or |\nameref| if the
% package \pkg{nameref} has been loaded to refer to anything which has
% been given a |\label| in either |aaa.tex| or the main document.
% You may declare any number of such external documents.
%
% If any of the external documents, or the main document, use the same
% |\label| then an error will occur as the label will be multiply
% defined. To overcome this problem |\externaldocument| has an optional
% argument \meta{prefix}. 
% If you declare |\externaldocument[A-]{aaa}|, then all
% references from |aaa| are prefixed by |A-|. So for instance, if a
% section of |aaa| had |\label{intro}|, then this could be referenced
% with |\ref{A-intro}|. The prefix need not be |A-|, it can be any
% string chosen to ensure that all the labels imported from external
% files are unique. Note however that the prefix is expanded and
% so should not contain commands that are not safe in this context. 
%
% As first suggested in Enrico Gregorio's |xcite| package, the current version
% also allows |\cite| to reference |\bibitem| in the external document.
% For compatibility with |xcite|, |\externalcitedocument| is made available
% as an alias for |\externaldocument|
% 
% Many packages have variant citation commands (natbib,
% biblatex,....) and the external document may or may not have used
% hyperref. Because of these differences the citation linking may not
% always work, it can be disabled by specifying \texttt{[nocite]} after the
% \meta{prefix}:
% \begin{verbatim}
% \externaldocument[][nocite]{aaa}
% \end{verbatim}  
% 
% The `document' referred to by the main argument \meta{document} is the file
% \file{document.aux} which must be somewhere on TeX's input path.
% Some packages (eg hyperref) really need to know the location of the
% final document rather than the aux file. By default this is assumed
% to be \file{document.pdf}. A package may redefine the command \cs{XR@ext} to
% change this default extension. However sometimes the final
% document may be in a position unrelated to the aux file, or the
% browser may not be able to find files at an arbitrary point in
% TeX's input path, so the final optional argument \meta{url} allows a full
% URL to the final document to be specified.
% \begin{verbatim}
% \externaldocument{aaa}[http://here.xxx.edu/this/path/to/aaa.pdf]
% \end{verbatim}
% 
% The package stores the url of the external document in the label data. If can
% e.g. be retrieved with the \pkg{refcount} package
% 
% \begin{verbatim}
% \usepackage{refcount,xr-hyper}
% \externaldocument{aaa}
% ...
% \getrefbykeydefault{intro}{url}{??} %prints aaa.pdf or ??
% \end{verbatim}
% 
% \pkg{xr-hyper} supports also the properties introduced in \LaTeX{} 2023-11-01.
% Here the url of the external document is stored in the \texttt{xr-url} property.
% 
% \begin{verbatim}
% \usepackage{xr-hyper}
% \externaldocument{aaa} %aaa contains \RecordProperties{intro}{page}
% ...
% \RefProperty{intro}{page}   %gives page number
% \RefProperty{intro}{xr-url} %gives aaa.pdf 
% \end{verbatim}
% 
%
% \MaybeStop{}
%
% \section{The macros}
%
%    \begin{macrocode}
%<*package>
%    \end{macrocode}
%
% Check for the optional argument.
%    \begin{macrocode}
\def\externaldocument{\@testopt\XR@cite{}}
\let\externalcitedocument\externaldocument
\def\XR@cite[#1]{\@testopt{\XR@[#1]}{}}
\def\XR@[#1][#2]#3{\@testopt{\XR@@{#1}{#2}{#3}}{#3.\XR@ext}}
%    \end{macrocode}
%
% \subsection{helper definitions}
% To test the second optional argument
%    \begin{macrocode}
\def\XR@@nocite{nocite}
%    \end{macrocode}
% Needed in the processing
%    \begin{macrocode}
\long\def\@gobblefour  #1#2#3#4{}
\long\def\@firstoffour #1#2#3#4{#1}
\long\def\@secondoffour#1#2#3#4{#2}
\long\def\@thirdoffour #1#2#3#4{#3}
\long\def\@fourthoffour #1#2#3#4{#4}
%    \end{macrocode}
% The url is added as fifth argument. The command used here is
% \cs{XR@addURL}. The command is more complicated as needed
% as it tries to handle also older documents with 
% \cs{newlabel}'s with two arguments. 
%    \begin{macrocode}
\def\XR@addURL#1{\XR@@dURL#1{}{}{}{}\\}
\def\XR@@dURL#1#2#3#4#5\\{%
     \unexpanded{{#1}{#2}{#3}{#4}}{\XR@URL}% 
  }%
%    \end{macrocode}
% 
% \subsection{Variables}
% 
% Default file extension:
%    \begin{macrocode}
\providecommand\XR@ext{pdf}
%    \end{macrocode}
%
% \subsection{Processing}
% Save the optional prefix. Start processing the first |aux| file.
% Version beta2 also added another improvement unrelated
% to the hyperref support. Olivier Michel pointed out that
% if the aux file was not on texinputs you could not always go
% \cs{externaldocument}{/some/path/to/file}
% specifically that worked if file.aux was a `simple'  document with
% one aux file, but if \cs{include} had been used, the `sub' aux files
% would not be found by xr in the remote directory.
% This version calls \cs{filename@parse} to get the directory name of the
% remote directory, which is then explicitly prepended to the names of
% any included aux files.
%  \changes{v5.06}{2020-05-10}{Remove leading and trailing sapces from
%    the filename (gh/2223) (xr)}
%    \begin{macrocode}
\def\XR@@#1#2#3[#4]{{%
  \makeatletter
  \def\XR@prefix{#1}%
   \def\XR@nocite{#2}%
   \ifx\XR@nocite\XR@@nocite
     \let\XR@bibcite\vadjust
   \else
     \let\XR@bibcite\bibcite
   \fi
  \def\XR@URL{#4}%
  \set@curr@file{#3}% 
  \filename@parse\@curr@file 
  \XR@next\@curr@file.aux\relax\\}}
%    \end{macrocode}
%
% Process the next |aux| file in the list and remove it from the head of
% the list of files to process.
%    \begin{macrocode}
\def\XR@next#1\relax#2\\{%
  \edef\XR@list{#2}%
  \XR@loop{#1}}
%    \end{macrocode}
%
% Check whether the list of |aux| files is empty.
%    \begin{macrocode}
\def\XR@aux{%
  \ifx\XR@list\@empty\else\expandafter\XR@explist\fi}
%    \end{macrocode}
%
% Expand the list of aux files, and call |\XR@next| to process the first
% one.
%    \begin{macrocode}
\def\XR@explist{\expandafter\XR@next\XR@list\\}
%    \end{macrocode}
%
% If the |aux| file exists, loop through line by line, looking for
% |\newlabel| and |\@input|. Otherwise process the next file in the
% list.
%  \changes{v5.06}{2020-05-10}{Add braces around the filename to
%    support filenames with spaces (gh/223) (xr)}
%  \changes{v7.01h}{2024-01-20}{Add braces around the filename to
%    support filenames with spaces (gh/223) similar to xr}
%    \begin{macrocode}
\def\XR@loop#1{\openin\@inputcheck{#1}\relax
  \ifeof\@inputcheck
    \PackageWarning{xr}{^^JNo file #1^^JLABELS NOT IMPORTED.^^J}%
    \expandafter\XR@aux
  \else
    \PackageInfo{xr}{IMPORTING LABELS FROM #1}%
    \expandafter\XR@read\fi}
%    \end{macrocode}
%
% Read the next line of the aux file.
%    \begin{macrocode}
\def\XR@read{%
  \read\@inputcheck to\XR@line
%    \end{macrocode}
% The |...| make sure that |\XR@test| always has sufficient arguments.
%    \begin{macrocode}
  \expandafter\XR@test\XR@line...\XR@}
%    \end{macrocode}
%
% Look at the first token of the line.
% If it is |\newlabel|, define \cs{r@}\meta{label}, ensure that it has
% five label data argument and add the url as the last one.  
% If it is |\@input|, add the
% filename to the list of files to process. 
% If it is |\bibcite|, call a |\bibcite|. 
% If it is |\new@label@record| add the url and then call it.
% Otherwise ignore.
% Go around the loop if not at end of file. Finally process the next
% file in the list.
% Make sure the arguments are handled outside the |\ifx| test,
%    \begin{macrocode}
\long\def\XR@test#1#2#3#4\XR@{%
  \let\XR@tempa\@gobblefour
  \ifx#1\newlabel
    \let\XR@tempa\@firstoffour
  \else\ifx#1\XR@bibcite
    \let\XR@tempa\@secondoffour
  \else\ifx#1\@input
     \let\XR@tempa\@thirdoffour
  \else\ifx#1\new@label@record
     \let\XR@tempa\@fourthoffour    
  \fi\fi\fi\fi
   \XR@tempa
    {%
     \expandafter\protected@xdef\csname r@\XR@prefix#2\endcsname{\XR@addURL{#3}}%     
    }%
    {\expandafter\bibcite\expandafter{\XR@prefix#2}{#3}}%
    {\edef\XR@list{\XR@list\filename@area#2\relax}}%
    {%
      \edef\next{\noexpand\new@label@record{\XR@prefix#2}{\unexpanded{#3}{xr-url}{\XR@URL}}}%
      \next
    }
  \ifeof\@inputcheck\expandafter\XR@aux
  \else\expandafter\XR@read\fi}
%    \end{macrocode}
%
%    \begin{macrocode}
%</package>
%    \end{macrocode}
%
% \Finale
%
